\documentclass{article}
\usepackage[utf8]{inputenc}

\title{Aula}
\author{Gabriel Cardoso 18.00477-6}
\date{March 2020}

\usepackage{natbib}
\usepackage{graphicx}

\begin{document}

\maketitle

\section{Axioma 1}
tempo para pegar e gravar variaveis
\sigma rec + \sigma Arm

\section{Axioma 2}
Os tempos nescessários para realizar operações elementares
\sigma +,-,x,/,<>


Ex:
y = y+1
rec: para ler 1
rec: para ler y
+: aperação
arm:gravar resultado

rsposta: \[2rec + "+" + arm \]

\section{Axioma 3}
Tempo nescessário para se chamar um método é constante : \sigma chmada
Tempo para retornar isso: \sigma retorno

\section{Axioma 4}
\sigma  arm

\section{Axioma 5}
Indices de Arrays
São armazenados em locais contiguos de memória
\sigma .

Ex: 

\[y = a[i] \]

Tempo: \[ 3(\sigma)r+(\sigma).+(\sigma)arm \]

\section{Modelo SImplificado}
Substituiu o tmepo pela quantidade de operações

\end{docuent}
