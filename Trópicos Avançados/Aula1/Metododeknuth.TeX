\documentclass{article}
\usepackage[utf8]{inputenc}

\title{Aula}
\author{Gabriel Cardoso 18.00477-6}
\date{March 2020}

\usepackage{natbib}
\usepackage{graphicx}

\begin{document}

\maketitle

\section{Axioma 1}
tempo para pegar e gravar variaveis

\section{Axioma 2}
Os tempos nescessários para realizar operações elementares
$\sigma +,-,x,/,<>$


Ex:
y = y+1
rec: para ler 1
rec: para ler y
+: aperação
arm:gravar resultado

rsposta: \[2rec + "+" + arm \]

\section{Axioma 3}
Tempo nescessário para se chamar um método é constante : $\sigma$ chmada
Tempo para retornar isso: $\sigma$ retorno

\section{Axioma 4}
$\sigma$  arm

\section{Axioma 5}
Indices de Arrays
São armazenados em locais contiguos de memória
$\sigma$ .

Ex: 

\[y = a[i] \]

Tempo: \[ 3(\sigma)r+(\sigma).+(\sigma)arm \]

\section{Modelo SImplificado}
Substituiu o tmepo pela quantidade de operações

\begin{table}
    \caption{Tabela verdade}
    \begin{tabular}{c|c|c|c|c|c|c}
        p & q & p$\rightarrow$q & $\neg$p & $\neg$q & 1$\wedge$2 & 1$\wedge$2$\rightarrow$q\\
        v & v & v & f & f & f & v\\
        v & f & f & f & v & f & v\\
        f & v & v & v & f & v & f\\
        f & f & v & v & v & v & v
    \end{tabular}

\end{table}

\end{document}
